\documentclass[12pt]{report}
\usepackage{array} %used for fix-width tables 
\usepackage{multirow} %needed for multiple rows or columns 
\usepackage{longtable} %for long tables 
\usepackage[table]{xcolor} %for coloring tables 
\begin{document}
\listoftables
\section{simple tables}

%this makes a simple matrix of information 
this is a simple table! 

\begin{tabular}{c c c}
asdf & bsdf & csdf \\
dsdf & esdf & fsdf \\
\end{tabular}

now with all the bells and whistles 
\begin{center}
\begin{tabular}[t]{||| c | c | c ||} %other options include l, c r (horizonal justification). In the square bracket,s you can do "t" (top vertical), "b" (bottom ertical), or don't put anything to get the default centering 
\hline 
asdf & bsdf & csdf \\
\hline \hline 
1 & 23 & 56 \\
\hline 
152 & 22 & 36 \\[1cm] %this [1cm] adds vertical space in a table
\hline 
\end{tabular}
\end{center}

with even more bells and whistles
\begin{center}
\begin{tabular}{*{3}{|c}|} %note how we can succinctly declare many columns 
\hline 
asdf & bsdf & csdf \\
\hline \hline 
1 & 23 & 56 \\
\cline{1-2} %partial line 
152 & 22 & 36 \\
\hline 
\end{tabular}
\end{center}

now with fixed length
\begin{center}
\begin{tabular}{||m{5em} | m{1cm}| m{1cm} ||} %other options are p (top), m (middle), and b(bottom) in terms of vertical alignment 
\hline 
asdf & bsdf & csdf \\
\hline \hline 
1 & 23 & 56 \\
\hline 
152 & 22 & 36 \\
\hline 
\end{tabular}
\end{center}

\section{merging things}
now with a merged column environment
\begin{center}
\begin{tabular}{|p{3cm}||p{3cm}|p{3cm}|p{3cm}|} 
\hline
\multicolumn{3}{|c|}{Title here}\\
\hline 
asdf & bsdf & csdf \\
\hline \hline 
1 & 23 & 56 \\
\hline 
152 & 22 & 36 \\
\hline 
\end{tabular}
\end{center}

now with a merged row environment
\begin{center}
\begin{tabular}{||| c | c | c ||} %note how the number of vbars is the number of bars on the output
\hline 
\multirow{2}{4em}{asdf} & bsdf & csdf \\
 & 23 & 56 \\ 
 \hline
 & 22 & 36 \\ 
 \hline
\end{tabular}
\end{center}

\pagebreak 

\section{long tables}
\begin{longtable}[c]{|c | c |}
\caption{test}\\
%this part will only be on the first page 
 \hline
 \multicolumn{2}{| c |}{Begin of Table}\\
 \hline
 Something & something else\\
 \hline
 \endfirsthead

%this part will be repeated on every page 
 \hline
 \multicolumn{2}{|c|}{Continuation of Table}\\
 \hline
 Something & something else\\
 \hline
 \endhead

%this part will be on the last page 
  \hline
 \endfoot

%this part will be at the end of every page 
 \hline
 \multicolumn{2}{| c |}{End of Table}\\
 \hline\hline
 \endlastfoot
  Lots of lines & like this\\
 Lots of lines & like this\\
 Lots of lines & like this\\
 Lots of lines & like this\\
 Lots of lines & like this\\
 Lots of lines & like this\\
 Lots of lines & like this\\
  Lots of lines & like this\\
 Lots of lines & like this\\
 Lots of lines & like this\\
 Lots of lines & like this\\
 Lots of lines & like this\\
 Lots of lines & like this\\
 Lots of lines & like this\\
  Lots of lines & like this\\
 Lots of lines & like this\\
 Lots of lines & like this\\
 Lots of lines & like this\\
 Lots of lines & like this\\
 Lots of lines & like this\\
 Lots of lines & like this\\
  Lots of lines & like this\\
 Lots of lines & like this\\
 Lots of lines & like this\\
 Lots of lines & like this\\
 Lots of lines & like this\\
 Lots of lines & like this\\
 Lots of lines & like this\\
  Lots of lines & like this\\
 Lots of lines & like this\\
 Lots of lines & like this\\
 Lots of lines & like this\\
 Lots of lines & like this\\
 Lots of lines & like this\\
  Lots of lines & like this\\
 Lots of lines & like this\\
 Lots of lines & like this\\
 Lots of lines & like this\\
 Lots of lines & like this\\
 Lots of lines & like this\\
 Lots of lines & like this\\
  Lots of lines & like this\\
 Lots of lines & like this\\
 Lots of lines & like this\\
 Lots of lines & like this\\
 Lots of lines & like this\\
 Lots of lines & like this\\
 Lots of lines & like this\\
  Lots of lines & like this\\
 Lots of lines & like this\\
 Lots of lines & like this\\
 Lots of lines & like this\\
 Lots of lines & like this\\
 Lots of lines & like this\\
  Lots of lines & like this\\
 Lots of lines & like this\\
 Lots of lines & like this\\
 Lots of lines & like this\\
 Lots of lines & like this\\
 Lots of lines & like this\\
 Lots of lines & like this\\
  Lots of lines & like this\\
 Lots of lines & like this\\
 Lots of lines & like this\\
 Lots of lines & like this\\
 Lots of lines & like this\\
 Lots of lines & like this\\
 Lots of lines & like this\\
  Lots of lines & like this\\
 Lots of lines & like this\\
 Lots of lines & like this\\
 Lots of lines & like this\\
 Lots of lines & like this\\
 Lots of lines & like this\\
  Lots of lines & like this\\
 Lots of lines & like this\\
 Lots of lines & like this\\
 Lots of lines & like this\\
 Lots of lines & like this\\
 Lots of lines & like this\\
 Lots of lines & like this\\
  Lots of lines & like this\\
 Lots of lines & like this\\
 Lots of lines & like this\\
 Lots of lines & like this\\
 Lots of lines & like this\\
 Lots of lines & like this\\
 Lots of lines & like this\\
 \end{longtable}

 \section{table objects}
 You can put a tabular environment inside a table environment, which allows you to position it and also label it 

 \begin{table}[h!]
    \centering
    \begin{tabular}{||| c | c | c ||} %note how the number of vbars is the number of bars on the output
        \hline 
        asdf & bsdf & csdf \\
        \hline \hline 
        1 & 23 & 56 \\
        \hline 
        152 & 22 & 36 \\
        \hline 
    \end{tabular}
    \caption{This is a table!}
    \label{table1}
\end{table}

As can be seen in Table \ref{table1}, we can reference it!

\section{table formatting}
\setlength{\arrayrulewidth}{0.5mm} %thickness of the lines 
\setlength{\tabcolsep}{18pt} %width of the columns 
\renewcommand{\arraystretch}{1.5} %spaces between rows 


\begin{center}
    \begin{tabular}{| c | c | c |} %note how the number of vbars is the number of bars on the output
        \hline 
        asdf & bsdf & csdf \\
        \hline \hline 
        1 & 23 & 56 \\
        \hline 
        152 & 22 & 36 \\
        \hline 
    \end{tabular}
\end{center}

time to color some tables! 
%now, if we want to color tables, we use the xcolor package 
{\rowcolors{1}{green!80!yellow!50}{blue!70!yellow!40} %left is odd, right is even. First parmaeter is when the coloring should start 
\begin{center}
    \begin{tabular}{| c | c | c |} %note how the number of vbars is the number of bars on the output
        \hline 
        asdf & bsdf & csdf \\
        \hline \hline 
        1 & 23 & 56 \\
        \hline 
        152 & 22 & 36 \\
        \hline 
        152 & 22 & 36 \\
        \hline 
        152 & 22 & 36 \\
        \hline 
        152 & 22 & 36 \\
        \hline 
        152 & 22 & 36 \\
        \hline 
    \end{tabular}
\end{center}


\end{document}