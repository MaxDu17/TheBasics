\documentclass[12pt]{report}

%for glossaries. Note: this will NOT work on mklatex because it requires a special compilation. Overleaf handles this quite well 
\usepackage{glossaries}
\makeglossaries 

\usepackage{imakeidx}
\makeindex

\makeindex[columns=3, title=Alphabetical Index, intoc]

%stylefiles are used https://www.overleaf.com/learn/latex/Indices for more freedom, but this is not implemented here for simplicity 

\begin{document}
\tableofcontents 
\pagebreak 

\index{the}
\index{fox} 
\index{dogs}

the quick brown fox jumped over the lazy dogs

%for nested indices. The base word is on the left and we can make more nests by going right
\index{polymerization ! spontaneous}
\index{polymerization ! slow ! days-long}

%now, we make a glossary entry 
\newglossaryentry{latex}
{
    name=latex,
    description={Is a markup language specially suited 
    for scientific documents}
}
\newacronym{gcd}{GCD}{Greatest Common Divisor}

To use, we just type out \Gls{latex}

\printglossaries
\printindex

% \printglossary[type=\acronymtype] %separates acronymns 

\end{document}