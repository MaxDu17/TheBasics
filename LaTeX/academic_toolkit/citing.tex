\documentclass[12pt]{report}
% \usepackage[]{biblatex}
\usepackage[backend=biber, style = ieee]{biblatex}
%style
    % verbose (everything at point of citation)
    % verbose-ibid yields a humanities-style citation with ibid 
        %for humanities citations, you also need to use enotez (see other tutorial)
    % numeric yields [1], [2], etc (default)
    % authoryear, authortitle
    % draft (displays file handles)
    % ieee
    % mla 
\addbibresource{ref.bib}


\begin{document}
\tableofcontents
\pagebreak 
\begin{refsection}

First, I will cite an article \cite{article}. Next, I will cite a book \cite{book}, and then a booklet \cite{booklet}. If I want to cite a chapter or page, try \cite{inbook} or \cite{inbookpages} respectively. 

What about manuals? Sure! \cite{manual} We also can do conference proceedings \cite{proceedings} and thesises \cite{thesis}. 

Random stuff goes here \cite{misc} and websites go here \cite{online}. Multiple citations look like this \cite{misc, online}

We can prefix citations and postfix citations. \cite[prefix][postfix]{manual}. This is helpful for footnote citations, as we will see below. 

Alternative citations can include \footcite{online} (helpful for verbose citations). We can wrap parenthesis around citations \parencite{online} (doesn't work here)

For author citation: \cite{online}. for short title: \citetitle{book}. for full title: \citetitle*{book} (doesn't make a difference here). For year: \citeyear{book}. For date: \citedate{online} (no difference here). For URL: \citeurl{online}

If you're feeling like full citations, try \fullcite{online} for an in-text full, and \footfullcite{online} for a footnote full. 

For a foolproof citation, use \autocite{online}, which does everything for you. 

Now, entries only appear in the bibliography if you cite them in the document. If you want to include things that are not explicitly cited, use \nocite{online}. Or, if you want to include everything, try \nocite{*}. 

\printbibliography[title = custom title for bibliography, heading=bibintoc]
%title: the title of the bibiiography
%type: which types should be displayed (article, book ,etc) (default is 'all')
%heading = bibintoc adds the bibiliography to the toc 
%all parameters are optional 

\end{refsection}
%you can add more \printbibliography commands later, as long as you encapsulate them with the refsection environments. If you only want one bibliography, then you can remove the refsection environments 

\end{document}