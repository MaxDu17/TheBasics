\documentclass[12pt]{report}
\usepackage{amsmath} %for most things 
\usepackage{amssymb} %for fonts 
\usepackage{mathtools} %for boxed within environment 
\usepackage{amsthm} %theorems and proofs 


\DeclareMathOperator*{\mysymbol}{OOGLABOOGLA} %define custom math symbol 
\begin{document}
\chapter{Math!}

\section{The basics}
This is math mode: $x = 2$. This is also math mode \(x = 3\). Display mode is blow 

$$ x = 2$$

or 

\[ x = 3\]

Fraction: $\frac{1}{2}$. 

Super and subscripts: $e_{subscript}^{superscript}$ 


\section{Different environments}
%the equation environment adds some more flexibility 
\begin{equation} \label{eq1} %this is how you label an equation 
    \begin{split}
    A & = 2 \\
    & = 3x
    \end{split}
\end{equation}

For longer equations 
\begin{multline*} %no numerical label this time 
p(x) = 3x^6 + 14x^5y + 590x^4y^2 + 19x^3y^3\\ 
- 12x^2y^4 - 12xy^5 + 2y^6 - a^3b^3
\end{multline*}

To align equations 
%you can use multiple alignment symbols as long as you are consistent 
\begin{align*} 
2x - 5y &=  8 \\ 
3x + 9y &=  -12
\end{align*} 

If you just want a few centered equations 
\begin{gather*} 
2x - 5y =  8 \\ 
3x^2 + 9y =  3a + c
\end{gather*}

\section{keywords}
All trig functions 

$$\max, \min, \ker, \exp, \deg, \gcd, \lg, \ln, \Pr, \sup, \det, \hom, \lim, \log, \arg$$

This is now $\mysymbol_2 = 2$

$$a \mathbin{BOB} b = 3$$ %for binary relationships 

$$a \mathrel{IS SMALLER THAN} b$$ %for relational operators 

\section{large operators}
$$\int_{1}^n$$

$$\rvert_0^1$$

$$\sum_{i = 1}^\infty$$

$$\prod_{i = 1}^n$$

$$\cup_{i = 1}^n$$

$$\cap_{i = 1}^n$$

$$\oint_1^n$$

$$\coprod_{i = 1}^n$$

\section{matrices}
inline matrix $\begin{smallmatrix} a & b\\ c & d \end{smallmatrix}$ 


plain matrix  %although all of these can also be used in-line 

$$\begin{matrix}
a & b & c \\
d & e & f \\
\end{matrix}$$

round matrix 
$$\begin{pmatrix}
a & b & c \\
d & e & f \\
\end{pmatrix}$$

square matrix (most common)
$$\begin{bmatrix}
a & b & c \\
d & e & f \\
\end{bmatrix}$$

curly matrix 
$$\begin{Bmatrix}
a & b & c \\
d & e & f \\
\end{Bmatrix}$$

piped matrix 
$$\begin{vmatrix}
a & b & c \\
d & e & f \\
\end{vmatrix}$$

double piped matrix 
$$\begin{Vmatrix}
a & b & c \\
d & e & f \\
\end{Vmatrix}$$

piecewise expression 

$$\begin{cases}
x^2 + 2 & x = 3\\
3 & x \neq 3
\end{cases}$$

\section{Equivalence relations}
$$x \not \leq 2$$ %how to negate any equivalence relation 

$$x \stackrel{a}{=} z$$ %stacking things 

\section{Brackets and stuff}
$$(a) [a] \{a\} \langle a\rangle |a| \|a\|$$ %different brackets 

Manual sizing 

$$\big( \Big( \bigg( \Bigg($$

For dynamic sizing

$$\left(\int_a^b\right)$$ %works with every type of bracket 

Asymmetrical dynamically sized bracket 

$$\left(\int_a^b\right.$$ %note the period 

\section{spacing}
%in increasing order
%the normal spacing falls between \! and \, 
\begin{align*}
f(x) &= x^2\! +3x\! +2 \\
f(x) &= x^2+3x+2 \\
f(x) &= x^2\, +3x\, +2 \\
f(x) &= x^2\: +3x\: +2 \\
f(x) &= x^2\; +3x\; +2 \\
f(x) &= x^2\ +3x\ +2 \\
f(x) &= x^2\quad +3x\quad +2 \\
f(x) &= x^2\qquad +3x\qquad +2
\end{align*}

%for a quick space, use 
$c~d$

%\thinmuskip \medmuskip \thickmuskip is used for spacing around operators. \medmuskip is for binary, and \thickmuskip is for relational 

\section{Display styles}

%these switches only work on one side of the equation 
\begin{align*} 
\displaystyle \int_a^bxdx &= \displaystyle 2 \\
\textstyle \int_a^bxdx &= \textstyle 2\\ %for in-line display 
\scriptstyle f(x) &= \scriptstyle 2\\ %superscript 
\scriptscriptstyle f(x) &= \scriptscriptstyle 2\\ %supersuperscript 
\end{align*}

\section{fonts}
$$3x^2 = 3 + 3$$ 
$$\mathnormal{3x^2 = 3 + 3}$$ %it's "bumpier", as you can see 
$$\mathrm{3x^2 = 3 + 3}$$
$$\mathit{3x^2 = 3 + 3}$$
$$\mathbf{3x^2 = 3 + 3}$$
$$\mathsf{3x^2 = 3 + 3}$$
$$\mathtt{3x^2 = 3 + 3}$$

\section{Proofs and Theorems}
\theoremstyle{definition} %boldface title, roman body (definitions, problems, and examples)
\theoremstyle{plain} %boldface title, italicized body (theorems, lemmas)
\newtheorem{mytheorem}{Theorem title}[section] %name, title, counter resets 
\newtheorem{myproof}{proof title}[mytheorem] %resets in every new theorem 

\theoremstyle{remark} %italicized title, roman body
\newtheorem*{remarks}{Remark title} %this gives an unnumbered environment 

%the "proof" environment exists by default 
\begin{mytheorem}
This is a test of what the theorem environment is capable of 
    \begin{myproof}
    Testing again! \qedsymbol 
    \end{myproof}
\end{mytheorem}

\begin{remarks}
test test
\end{remarks}

%to do a proof, just make a special theorem environment and use the \qedsymbol to end your proof 
\end{document}