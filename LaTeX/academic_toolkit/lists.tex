\documentclass[12pt]{report}
\usepackage{enumitem} %for more customizable lists 

\begin{document}
\section{easy lists}

Here's a bulletted list 
\begin{itemize}
\item this is a list 
\item more items 
\item even more items 
\end{itemize}

Here's a list of numbers 
\begin{enumerate}
\item this is a numbered list
\item another one 
\end{enumerate}

Here's a list of descriptoins 
\begin{description}
\item[test test] another textu which also we want to start looking like lorem ipsum but it's much easilr because we can just ranambel adn we dont's eve nacarte ew;i awioehksajdlkjn  
\item[another one] here's another one and as you can see, 
\end{description} 

\section{customization}
%customize your bullets 
\begin{itemize}
\item[!!] this is a list 
\item[hello] more items 
\item[] even more items 
\end{itemize}

%nested lists. Lists are limited to four items deep, but enumitem allows an arbitrary number 
\begin{enumerate}
\item this is in the outer list  
    \begin{itemize}
    \item hello!
    \item hello! 
    \end{itemize}
\item here we are again! 
\end{enumerate}

%demonstration of how the numbering style changes 
\begin{enumerate}
\item this is in the outer list  
    \begin{enumerate}
    \item hello!
    \item hello! 
    \end{enumerate}
\item here we are again! 
\end{enumerate}

\begin{itemize}
\item this is in the outer list  
    \begin{itemize}
    \item hello!
    \item hello! 
    \end{itemize}
\item here we are again! 
\end{itemize}

%to change the different levels of bullets, we use the specical variables and counters
\renewcommand{\labelitemi}{QwAcK}
\renewcommand{\labelitemii}{smallerQwack}
\renewcommand{\labelitemiii}{smallerQwack}
\renewcommand{\labelitemiv}{smallerQwack}

\renewcommand{\labelenumi}{QwAcK}
\renewcommand{\labelenumii}{smallerQwack}
\renewcommand{\labelenumiii}{smallerQwack}
\renewcommand{\labelenumiv}{smallerQwack}

\begin{itemize}
\item this is in the outer list  
    \begin{itemize}
    \item hello!
    \item hello! 
    \end{itemize}
\item here we are again! 
\end{itemize}

we are at \theenumi or \roman{enumi} if you're interested in roman numbers
%use enumi, enumii, enumiii, enumiv for the counters 

\section{using \texttt{list}}
\newcounter{boxlblcounter}  
\newcommand{\makeboxlabel}[1]{\fbox{#1.}}
\newenvironment{boxlabel}
  {\begin{list}
    {\arabic{boxlblcounter}} %what each label should be 
    {\usecounter{boxlblcounter} 
     \setlength{\labelwidth}{3em} %space between number and text
     \setlength{\itemsep}{2pt} %space between items
     \setlength{\leftmargin}{1.5cm} %self explanatory, reference from edge of text 
     \setlength{\rightmargin}{2cm}
     \let\makelabel=\makeboxlabel %this command intercepts the default function and remaps it. this allows the box to be drawn around it 
    }
  }
{\end{list}}

\begin{boxlabel}
\item asdf
\item asdf
\item asdf
\end{boxlabel}

\section{enumitem package demo}
the whole deal with the boxlabel shows how hacky a custom list can be. Therefore, there is an easier package out there that prevents the complications. 

\newlist{myitems}{enumerate}{3} %name, type, depth. Depth can be an arbitrary number! 
\setlist[myitems, 1] %setting level 1 functionality. Remove the number to set everything 
{label=WW\arabic{myitemsi}., %this is a default counter
leftmargin=\parindent, %some parameters 
rightmargin=10pt
}
%you can continue defining the other two depths, but we will not do it here for brevity 

\begin{myitems}
\item asdf asdf asdf asdf asd 
\item test again 
\end{myitems}

\end{document}