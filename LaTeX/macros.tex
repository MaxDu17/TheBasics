\documentclass[12pt]{report}
\usepackage{xargs} %for xargs demo 

\begin{document}
\section{new commands}

\newcommand{\mycommand}{hello world!} %can be anything, not just text 


\mycommand

%with required arguments 
\newcommand{\myargumentcommand}[2]{\textbf{#1}, lol it's #2}

\myargumentcommand{hello world again!}{cats!}

%with optional parameters 
\newcommand{\myoptionalcommand}[2][cats]{\textbf{#2}, lol it's #1} %the first default is "cats"

\myoptionalcommand{hi!}

\myoptionalcommand[dogs]{hi!}

%can change an existing command
\renewcommand{\textit}[1]{lol you thought that #1}
\textit{dogs are people}

\section{xargs} %an easier way of making macros with optional arguments
\newcommandx{\hellodog}[3][1 = bob, 3 = joe, usedefault=@]{Hello #1, #2, #3}
%the usedefault means that if you put a @ in the square bracket, you use the default values in the declaration. By default, the usedefault symbol is an empty string so you can do []
\hellodog[steve]{max} %use default for "joe", override bob with steve, and use required "max"

\hellodog[@]{max}[] %note how the empty string replaces "joe", and the @ indicates that we want "bob" for the first argument 


\section{new environments}

\newenvironment{dogworld}{\texttt{Welcome to dog world!}}{\texttt{Goodbye from dog world!}}

\begin{dogworld}

Some text here 

\end{dogworld}

\newenvironment{catworld}[1]{\texttt{Welcome to cat world, #1!}}{\texttt{Goodbye from cat world!}}
%you can override an existing environment by using \renewenvironment 

\vspace{2em}

\begin{catworld}{Max}

Some text here 

\end{catworld}




\end{document}