\documentclass[12pt]{report}

\begin{document}
\section{basic counters}

\newcounter{mycounter} %default is zero
\newcounter{mycounter2}[section] %resets every section 
\setcounter{mycounter}{23} 
\addtocounter{mycounter}{2}
\addtocounter{mycounter}{-6}
\stepcounter{mycounter} %steps by one 
\stepcounter{mycounter2} 

The counter is currently at \themycounter

%other ways of showing what the counter can do 
\arabic{mycounter}

\roman{mycounter}

\Roman{mycounter}

\Alph{mycounter}

\alph{mycounter}

\fnsymbol{mycounter2} 

% \value{mycounter} %returns a latex number for use in other commands
\addtocounter{mycounter}{\value{mycounter}} %double your counter value 

%reference a counter, like you might for a picture or a table 
\refstepcounter{mycounter}
\label{test}

As we can see in \ref{test}, we can reference a counter 

\section{Linking counters}
\newcounter{outercounter}
\newcounter{innercounter}
\newcounter{outeroutercounter}
\counterwithin{innercounter}{outercounter}
\counterwithin{outercounter}{outeroutercounter} %you can recursively nest! 

\stepcounter{innercounter}
\theinnercounter %note how this is redefined to include both counters 

\stepcounter{innercounter}
\noindent \theinnercounter

\stepcounter{innercounter}
\noindent\theinnercounter

%now, when we step the outer counter, we see that the inner counter resets 
\stepcounter{outercounter}
\noindent\theinnercounter

%we unlink one of our dependencies 
\counterwithout{outercounter}{outeroutercounter} 
\stepcounter{outercounter}
\noindent\theinnercounter

\section{Existing counters}

%the counter names are these, minus the "the"
\thepart

\thechapter 

\thesection

\thesubsubsection 

\theparagraph 

\thesubparagraph 

\thefigure 

\theequation

\thetable

\thefootnote

\theenumi

\theenumii

\theenumiii

\theenumiv
\end{document}

